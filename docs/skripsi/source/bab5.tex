%!TEX root = ./template-skripsi.tex
%-------------------------------------------------------------------------------
%                            	BAB IV
%               		KESIMPULAN DAN SARAN
%-------------------------------------------------------------------------------

\chapter{KESIMPULAN DAN SARAN}

\section{Kesimpulan}
Berdasarkan hasil implementasi dan pengujian fitur yang telah dirancang, maka diperoleh kesimpulan sebagai berikut

\begin{enumerate}
	\item Perancangan aplikasi \textit{admin console} untuk manajemen dan visualisasi data hasil pengindeksan \textit{search engine} yang telah dibuat pada penelitian \citep{lazu} yang dirancang menggunakan metode \textit{scrum}.
	\item Berdasarkan pengujian \textit{unit testing} yang dilakukan oleh tim pengembang, aplikasi berjalan dengan baik. Berdasarkan hasil pengujian UAT terdapat feedback untuk suatu fitur yaitu \textit{crawling} pada bagian peta situs. Pada fitur ini terdapat masalah berupa performa yang tidak bagus sehingga disarankan untuk penyederhanaan fitur.
\end{enumerate}

\section{Saran}

Adapun saran untuk penelitian selanjutnya adalah:

\begin{enumerate}
	\item Penggunaan memori dari aplikasi yang meningkat secara signifikan dari waktu ke waktu menyebabkan \textit{server} dapat mengalami kehabisan \textit{memory}. Perlu diadakan penulisan kode yang lebih efisien dalam hal penggunaan memori.
	\item Pada penelitian dijumpai kasus kasus kecil yang terabaikan yang menyebabkan proses berjalannya aplikasi menjadi terhambat. Hal ini dapat dicegah dengan melakukan pengujian kepada kasus kecil yang mungkin terlewat.
\end{enumerate}


% Baris ini digunakan untuk membantu dalam melakukan sitasi
% Karena diapit dengan comment, maka baris ini akan diabaikan
% oleh compiler LaTeX.
\begin{comment}
\bibliography{daftar-pustaka}
\end{comment}