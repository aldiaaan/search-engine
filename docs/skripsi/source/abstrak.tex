\chapter*{\centering{\large{ABSTRAK}}}

\begin{spacing}{1}
	\textbf{ALDIAN ASMARA}. Perancangan \emph{User Interface Search Engine} dengan \textit{Admin Console} Untuk Manajemen dan Visualisasi Data Hasil Pengindeksan Search Engine. Skripsi. Fakultas Matematika dan Ilmu Pengetahuan Alam, Universitas Negeri Jakarta. 2024. Di bawah bimbingan Muhammad Eka Suryana, M.Kom dan Med Irzal, M.Kom.
	\newline
	\newline
	Mesin pencari merupakan sebuah program komputer yang berfungsi untuk membantu pengguna dalam menemukan informasi dengan kata kunci tertentu. Pada penelitian \citep{lazu} telah dirancang sebuah arsitektur \textit{search engine} dengan mengintegrasikan \textit{web crawler}, algoritma \textit{page ranking} dan \textit{document ranking}. Penelitian tersebut memiliki beberapa kekurangan yaitu tidak adanya \textit{admin console} untuk manajemen dan visualisasi data hasil pengindeksan \textit{search engine} yang telah dibuat. Penelitian ini memiliki tujuan untuk menyediakan suatu cara bagi pengguna untuk mengakses \textit{search engine} yang telah dibuat beserta \textit{admin console} untuk manajemen dan visualisasi data hasil pengindeksan \textit{search engine}. Informasi pendukung untuk melakukan penelitian ini berasal dari studi literatur jurnal-jurnal terkait dan diskusi yang diadakan peneliti dengan \textit{stakeholder}. Proses pengembangan yang digunakan dalam penelitian ini menggunakan metode \textit{Scrum} dengan menggunakan teknologi \textit{Python} dan \textit{Javascript}. Hasil akhir dari penelitian ini adalah sebuah  \textit{user interface} search engine \textit{admin panel} untuk manajemen dan visualisasi data hasil pengindeksan \textit{search engine} yang telah dibuat.
	\newline
	\newline
	\noindent \textbf{Kata kunci}:  \textit{search engine}, aplikasi, \textit{admin console}, \textit{scrum}
\end{spacing}

